\chapter{Testing}
\label{ch-testing}

\section{Round Trip Time and Throughput}
\label{sec:rt-troughput}

In order to test the Response Time for a LKL device we have chosen the router and built a simple topology shown below:
\fig[scale=0.5]{src/img/throughput.png}{img:throughput}{Topology for testing Response Time and Throughput}
\begin{center}
  \begin{table}[htb]
  \begin{center}
  \begin{tabular}{| l | l | l | l |}
    \hline
      Device Name & Interface & IP & Port \\ \hline
      Router0 & eth0 & 191.168.100.1/24 & x \\ \hline
      Hub0 & x & x & 50001\\ \hline
    \hline
  \end{tabular}
  \end{center}
  \caption{Topology description}
  \label{table:resp-time}
  \end{table}
\end{center}


For injecting traffic in the topology a bridge was connected to the hub and a route to the router's interface was added via tap0. In order for the bridge to connect to the hub, a configuration file is passed as a command line argument when running it for specifying the port and system address where the hub runs.
The  configuration file is: 
\lstset{language=TeX, caption=Bridge configuration file}
\lstinputlisting{src/code/bridge/bridge-config}
For starting the brodge, root permissions are needed in order to create the tap interface.
The commands for setting the bridge, after starting it are:
\lstset{language=TeX,caption=Commands for configuring the bridge,label=test-bridge-config}
\begin{lstlisting}
Setting the tap interface up: sudo ip link set tap0 up
Setting an IPv4 address on the new created tap interface: sudo ip addr add 192.168.100.100/24 dev tap0
Adding a new route to the router interface through tap0 interface: sudo ip route add 192.168.100.0/24 dev tap0
\end{lstlisting}

The next step is to inject traffic into the topology via the bridge(tap0) using ping.
For two transmissions of 1000 packets using ping (in)the results are:
\begin{lstlisting}
$sudo ping -f 192.168.100.1 -c 1000 -s 1400
PING 192.168.100.1 (192.168.100.1) 1400(1428) bytes of data.
                
--- 192.168.100.1 ping statistics ---
1000 packets transmitted, 1000 received, 0% packet loss, time 13771ms
rtt min/avg/max/mdev = 35.973/114.888/200.100/34.401 ms, pipe 16, ipg/ewma 13.785/128.326 ms

$ sudo ping -f 192.168.100.1 -c 1000 
PING 192.168.100.1 (192.168.100.1) 56(84) bytes of data.
               
--- 192.168.100.1 ping statistics ---
1000 packets transmitted, 1000 received, 0% packet loss, time 13538ms
rtt min/avg/max/mdev = 6.929/112.462/196.023/34.576 ms, pipe 15, ipg/ewma 13.552/105.503 ms
\end{lstlisting}
Let us now interpret the results of ping command:
As it can be seen, the transmission time does not change dramatically with the size of packets, suggesting that the bottleneck is not the time to process a packet of a given size but maybe the necessary time for the OS to start the thread for the LKL network interface.
\begin{itemize}
\item \textbf{Round Trip Time} is defined as the length of time it takes for a packet to be sent plus the length of time it takes for an acknowledgement of that packet to be received.
As it can be seen from ping output 

\textbf{rtt min/avg/max/mdev = 6.929/112.462/196.023/34.576 ms}
for a 56B packet length (the default packet size for ping) and 

\textbf{rtt min/avg/max/mdev = 35.973/114.888/200.100/34.401 ms}

for a 1400B packet size. There is no significant difference for the average,maximum RTT time or the standard deviation\footnote{\url{http://en.wikipedia.org/wiki/Standard_deviation}}, the only difference being in the minimum RTT for a packet.  
\item \textbf{Inter-Packet Gap} is the idle time between two subsequent packet transmissions which gives the medium enought time to stabilize and network devices time to process the packet;

\item \textbf{Exponential Weighted Moving Average}\footnote{\url{http://en.wikipedia.org/wiki/Moving_average_(finance)#Exponential_moving_average}} During operations, the effective idletime is measured using an exponential weighted moving average (EWMA), which considers recent packets to be exponentially more important than past ones. 

\item \textbf{Throughput} is defined as the average rate of successful message delivery over a communication channel. Its measurement unit is either bps (bits per second) or packets per second.
I have decided to compute the \textbf{throughput} based on the RTT as 1000/(2*RTT) which issues a throughput value of \approx 37 packets per second. 

\section{Scalability}
\label{sec:scalability}

\section{Complex Topologies}
\label{sec:complex-top}
