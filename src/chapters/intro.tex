\chapter{Introduction}
\label{chapter:intro}

\section{Project Description}
\label{sec:proj}

\project is an application built using Linux kernel library (a description of LKL can be found in \secref{sec:lkl}) with the purpose of implementing network devices such as router, switch, nat, firewall and hub by reusing the TCP/IP stack from the Linux kernel so that in the end every device will offer a full TCP/IP stack. Since being built using LKL, \project will benefit from the enhancements made in the TCP/IP stack from the Linux kernel.

The purpose of this project is firstly to simulate very complex topologies and, secondly to not be isolated from non-LKL world.
It was intended for \project to be able to communicate with non-LKL processes so that it can benefit from the usage of some tools. Throughout the \project development this feature was of a great aid for testing the behaviour of devices(ping and wireshark were frequently used). 

Besides the devices implemented by using LKL, two more devices (built as standard applications) were needed: a bridge for the purpose of connecting LKL processes with non-LKL ones and a hub for connecting \project devices. Also the hub is an entry point for non-LKL processes into the LKL world through the use of tun/tap interfaces.  

Network topologies can be defined using configuration files and, but not mandatory, a GUI. The entire topology is controlled by a hypervisor which boots the devices, saves changes done in the topology and communicates with the devices in order to send them information about the topology.

Every LKL network device is a new LKL instance which has a reserved memory space of 16MB RAM.
Implementation details, configuration file examples and the associated syntax are given in \chapref{chapter:impl} and in Appendix  \ref{ch:config-file}.
\subsection{Project Scope}
\label{sub-sec:proj-scope}
Simulators come with a great advantage over real network devices that is the possibility of constructing very complex topologies in a short time at a much lower cost. Simulators have a very similar behaviour to real-world topologies thus being a perfect tool for educational and research purposes. Obviously there are some disadvantages as well, such as lower performance levels than real network devices and the fact that after all they are only simulating topologies.

\project is intended to be used for \textbf{educational} purposes, as a tool for networking-related courses in faculty and, also for \textbf{research} purposes hence the need for a command line interface and a graphical user interface so that configuring the devices could be easyly done and the focus could be on observing topology behaviour and creating new scenarios.

The way \project is built enables researches to test new network protocols by implementing them in the Linux kernel or to test the impact some changes in current protocols could have. If any of these turn to be feasible the effort necessary to integrate them with the Linux kernel would be minimum. Also, being able to communicate with other processes through the use of tun/tap interface, might facilitate an easier testing process and looking from the educational point of view might allow an accurate investigation of traffic and thus a better understanding of networking protocols and behaviours.  

\subsection{Project Objectives}
\label{sub-sec:proj-objectives}
Some objectives have been defined for \project:
\begin{itemize}
\item Simulate network devices (host, switch, router, nat, firewall) using LKL and afterwards simulate topologies having a great number of network devices (even thousands of network devices)
%\item Even though not a stated objective, \project has proven once again that when needing to reuse some already implemented features from the Linux kernel, LKL turns out to be a feasible solution.
\item Devices from the topology should be able to communicate with non-LKL processes
\item \project should be easy to use
\end{itemize}

If, and how these objectives were met during the \project development is the subject of this bachelor thesis.
\section{Related Work}
\label{sec:proj-related}

\subsubsection{Dynamips}
Dynamips is an emulator written to emulate Cisco routers which runs on Linux, Windows and Mac OS X. It started as a project to emulate a Cisco 7200 but now supports  Cisco 3600, 3700 and 2600 series.
Being an emulator, Dynamips devices act like real devices but at a lower speed; the best performance one can get on a host machine is 1kpps\footnote{As stated on the project's homepage \url{http://www.ipflow.utc.fr/index.php/Cisco_7200_Simulator}}.
Below are listed some characteristics of Dynamips:
\begin{itemize}
\item Dynamips requires a Cisco IOS image in order to emulate the devices 
\item The number of network devices Dynamips can emulate is limited as it requires high memory and has a high CPU load
\item It can only emulate routers and no other devices and the routers emulated are exclusively Cisco routers
\item The source code is licensed under the GNU GPL which grants rights to modify and redistribute the software
\end{itemize}

\subsubsection{Packet Tracer}
Packet Tracer is a Cisco proprietary network simulation program used as a teaching and learning tool. 
It allows users to create network topologies, configure devices, inject packets, and simulate a network with multiple visual representations.

Since it is a simulator, Packet Tracer utilizes only a small number of features found within the actual 
hardware running a current Cisco IOS thus being appropriate for learning purposes but not for realistic network simulations.
It is a closed source program so that no modifications can be done to the devices.

\subsubsection[Network Simulator ns-2]{Network Simulator ns-2\footnote{\url{http://www.isi.edu/nsnam/ns/}}}

ns-2 is a discrete event simulator targeted to research. It provides substantial support for simulation of TCP, routing and multicast protocols over wired and wireless networks. It is still an ongoing project.
